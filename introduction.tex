\doublespacing

\chapter{INTRODUCTION}
% \addcontentsline{toc}{chapter}{Introduction}
This is where you begin your thesis. Be sure to indent all paragraphs 1/2 inch. Do not put extra spacing between paragraphs. In addition, paragraphs should be formatted so that there is zero added space above/below them. Justify on the left side only. Consult the style manual approved by the faculty in your program to determine appropriate actions on widow and orphan issues \cite{article-minimal}.

\section{Spacing}
Double space the body of the text (except the abstract). Single spacing may be required for footnotes or quotations of five lines or more, and may be used for table headings and figure captions. In addition, single spacing is acceptable for subheadings in the Table of Contents and in the Acknowledgement Page if this enables these sections to be one page. Finally, references may be single spaced within the reference and double spaced between references. Consult the style manual approved by the faculty in your program for appropriate reference format. Also, note that after the primary heading on this page, and before each secondary heading, there are blank lines. This is appropriate spacing format for these types of headings.

\subsection{Subsection Test}
Lorem ipsum dolor sit amet, consectetur adipisicing elit, sed do eiusmod
tempor incididunt ut labore et dolore magna aliqua. Ut enim ad minim veniam,
quis nostrud exercitation ullamco laboris nisi ut aliquip ex ea commodo
consequat. Duis aute irure dolor in reprehenderit in voluptate velit esse
cillum dolore eu fugiat nulla pariatur. Excepteur sint occaecat cupidatat non
proident, sunt in culpa qui officia deserunt mollit anim id est laborum.

\subsubsection{Subsubsection Test}
Lorem ipsum dolor sit amet, consectetur adipisicing elit, sed do eiusmod
tempor incididunt ut labore et dolore magna aliqua. Ut enim ad minim veniam,
quis nostrud exercitation ullamco laboris nisi ut aliquip ex ea commodo
consequat. Duis aute irure dolor in reprehenderit in voluptate velit esse
cillum dolore eu fugiat nulla pariatur. Excepteur sint occaecat cupidatat non
proident, sunt in culpa qui officia deserunt mollit anim id est laborum.

\section{Headings}
Headings are essential for dividing the body of the thesis, and a standard format is required by the Graduate College. This format may be an exception to the style manual approved by the faculty in your program, but you are to follow the Thesis Guide.

Headings should be descriptive, focus attention on distinctive sections, and thus enable a quick targeting of salient information addressed in the thesis. Depending on the nature of the subject, more than one level of heading may be appropriate. It is vital that there is a consistency in placement and other aspects of formatting headings that divide the text.

Start main (primary) headings on a new page. These primary headings should be centered, bold, upper case, and separated from the text that follows by extra space (blank line). Secondary headings should follow a blank line (double-spacing), be placed at the left margin, bold, and capitalize only the first letter of words. Tertiary headings will be placed as the first word(s) of the paragraph of that section, indented, bolded, first letter of words capitalized, and followed by a period. Fourth level headings should be indented, underlined, with the first letter of words capitalized, followed by a period; fifth level heading should be indented, italicized with the first letter of words capitalized, followed by a period. The first sentence of the paragraph will then follow on the same line for 3 rd - 5 th level headings. The format of primary, secondary, and tertiary headings is modeled in this Thesis Guide. Make sure your Table of Contents matches these heading types.

It is not acceptable to have just one subheading under a larger heading. For example, if you are to use secondary headings under a primary heading, there must be two or more headings. This would be analogous to an outline that has an ``A'' but no ``B.''
